% PLEASE USE THIS FILE AS A TEMPLATE
% Check file iosart2x.tex for more examples

% add. options: [seceqn,secthm,crcready]
\documentclass[aac]{iosart2x}

%\usepackage{dcolumn}

%%%%%%%%%%% Put your definitions here


%%%%%%%%%%% End of definitions

\pubyear{0000}
\volume{0}
\firstpage{1}
\lastpage{1}


\begin{document}

\begin{frontmatter}

%\pretitle{}
{\centering \title{Using Tableau as Tool for Time Series Visualization in the MIMIC III Database 
%- Lessons Learned/A (technical) Case Report
}}
\runtitle{Tableau as Tool for Time Series Visualization}
%\subtitle{}

% For one author:
%\author{\inits{N.}\fnms{Name1} \snm{Surname1}\ead[label=e1]{first@somewhere.com}}
%\address{Department first, \orgname{University or Company name},
%Abbreviate US states, \cny{Country}\printead[presep={\\}]{e1}}

% Two or more authors:
\author[A]{\inits{K.}\fnms{Karl} \snm{GOTTFRIED}\ead[label=e1]{k.gottfried@uke.de}%
\thanks{Corresponding Author, Karl Gottfried, Applied Medical Informatics, Hamburg, University Hospital Hamburg-Eppendorf, Martinistraße 52, 20246 Hamburg, Germany;  \printead{e1}.}},
\author[A]{\inits{S.}\fnms{Sylvia} \snm{NÜRNBERG}\ead[label=e2]{second@somewhere.com}}
and
\author[A]{\inits{F.}\fnms{Frank} \snm{ÜCKERT}\ead[label=e3]{third@somewhere.com}}
\address[A]{Applied Medical Informatics, , Germany, \orgname{University Hospital Hamburg-Eppendorf},
\cny{Germany}\printead[presep={\\}]{e1}}
\address[B]{Department first, \orgname{University or Company name},
Abbreviate US states, \cny{Country}\printead[presep={\\}]{e2,e3}}


\begin{abstract}
The meaningful visualization of clinical records can help to facilitate interpretation and exploration among physicians scientists or clinicians, especially with extensive data sets. The MIMIC-III database is an intensive care database, which documents in detail the course and treatment of more than 40,000 patients and can be used as the basis for visualization tools in the medical field. In the present work, we developed two exemplary dashboards with Tableau software, that display time series for individual patients and for a selected cohort on the MIMIC-Database. Once created, dashboards can be adapted for specific queries by implementing of dynamic parameters and thus reused. This enables the interactive visualization and exploration of clinical data for non-technical users and also allows them to create customized dashboards.
\end{abstract}

\begin{keyword}
\kwd{Tableau}
\kwd{Data Visualization}
\end{keyword}

\end{frontmatter}

%%%%%%%%%%% The article body starts:

\section{Introduction}\label{s1}

\subsection{Background}\label{s1.1}
\noindent Tableau is a software platform designed for visual exploration and was initially developed and used for
data analysis outside of the medical field ~\cite{Ko.2017} [Literatur Tableau seite]. Due to the continuously growing
amounts of data and the desire to be able to use this information more efficiently in a clinical and
scientific context, the medical field has also become a possible application for data visualization tools
such as Tableau ~\cite{Ko.2017} [weitere Literatur]. A clinical data source that has already been used to visualize patient data is the MIMIC-III (Medical Information Mart for Intensive Care) database, a freely accessible critical care database~\cite{Festag.2019,Lee.2016,Johnson.2020,Johnson.2016}. The third version of the MIMIC data set (MIMIC III) contains extensive clinical parameters of more than 40,000 patients who were admitted and treated in a medical intensive care units of the Beth Israel Deaconess Medical Center (Boston, Massachusetts, USA) between 2001 and 2012. Data classes ranges from clinical measurements like nurse-verified physiological measurements or laboratory test results, administrative
information like Current Procedural Terminology (CPT) codes and Diagnosis-Related Group (DRG) codes, patient characteristics like Demographic detail and dates of death to free-text interpretations of imaging studies provided by the radiology department~\cite{Johnson.2020,Johnson.2016}. The data was first deidentified in accordance with Health Insurance Portability and Accountability Act (HIPAA) standards before it was incorporated into the database. Furthermore researcher who wants to have access to the data have to pass an online course and accept a data use agreement before they are granted access by the curators ~\cite{Johnson.2020,Johnson.2016}.


\subsection{Objective and Requirements}\label{s1.2}



\section{State of the art}\label{s2}


\section{Concept}\label{s3}


\section{Implementation}\label{s4}

\section{Discussion and Outlook}\label{s5}

\section{Conclusion}\label{s6}
Using Tableau Software, we developed two interactive dashboards to display time-dependent variables. The approach presented can be expanded or adapted depending on the context and thus supplement already established analysis methods for specific questions. An implementation of the presented dashboard is available at ...



%\begin{figure}[t]
%\includegraphics{}
%\caption{Figure caption.}\label{f1}
%\end{figure}

%\begin{table*}
%\caption{} \label{t1}
%\begin{tabular}{lll}
%\hline
%&&\\
%&&\\
%\hline
%\end{tabular}
%\end{table*}

%%%%%%%%%%% The bibliography starts:

%%%%%%%%%%%%%%%%%%%%%%%%%%%%%%%%%%%%%%%%%%%%%%%%%%%%%%%%%%%%%
%%                  The Bibliography                       %%
%%                                                         %%
%%  ios1.bst will be used to                               %%
%%  create a .BBL file for submission.                     %%
%%                                                         %%
%%                                                         %%
%%  Note that the displayed Bibliography will not          %%
%%  necessarily be rendered by Latex exactly as specified  %%
%%  in the online Instructions for Authors.                %%
%%                                                         %%
%%%%%%%%%%%%%%%%%%%%%%%%%%%%%%%%%%%%%%%%%%%%%%%%%%%%%%%%%%%%%


\nocite{*} 
% if your bibliography is in bibtex format, use those commands:
\bibliographystyle{ios1}           % Style BST file.
\bibliography{bibliograpy_tableau.bib}        % Bibliography file (usually '*.bib')

% or include bibliography directly:
%\begin{thebibliography}{0}
%\bibitem{r1} F. Author, Information about cited object.
%
%\bibitem{r2} S. Author and T. Author, Information about cited object.
%\end{thebibliography}

\end{document}
