Tableau is a software platform designed for visual exploration and was initially developed and used for data analysis outside of the medical field ~\cite{Tableau.2021}. Due to the continuously growing amounts of data and the desire to use this information more efficiently in a clinical and scientific context, the medical field has also become a possible application for data visualization tools like Tableau ~\cite{Ko.2017} [weitere Literatur]. 
Furthermore, Sitting et al. states that there is no standard for the visual representation of laboratory values and that every commercially available electronic health record offers its own visualization solution \cite{Sittig.2015}. According to Sitting et al. not only the simple and intuitive interpretation of laboratory values, but can also lead to misinterpretations, whereby the visualization of clinical data is also relevant for patient safety and underlines the importance of correct visualizations in clinical practice.
A clinical data source that has already been used to visualize patient data is the MIMIC-III (Medical Information Mart for Intensive Care) database, a freely accessible critical care database~\cite{Festag.2019,Lee.2016,Johnson.2020,Johnson.2016}. The third version of the MIMIC data set (MIMIC III) contains extensive clinical parameters of more than 40,000 patients who were admitted and treated in medical intensive care units of the Beth Israel Deaconess Medical Center (Boston, Massachusetts, USA) between 2001 and 2012. Collected data classes ranges from clinical measurements like nurse-verified physiological measurements, laboratory test results or administrative information like Current Procedural Terminology (CPT) codes and Diagnosis-Related Group (DRG) codes to patient characteristics like Demographic detail and dates of death or free-text interpretations of imaging studies provided by the radiology department ~\cite{Johnson.2020,Johnson.2016}. The collected data was first de-identified in accordance with Health Insurance Portability and Accountability Act standards before it was incorporated into the database. Furthermore to protect the privacy of the patients, researcher have to pass an online course and accept a data use agreement before they get access to the database ~\cite{Johnson.2020,Johnson.2016}. A demo data set with 100 patients is provided without restriction for test purposes. 
The present work uses the MIMIC Data as a clinical data set to enable specific queries with the help of Tableau, such as the visualization of time series data for certain patients or the selection of cohorts based on relevant inclusion and exclusion criteria via a visually explorable representation. In this way, for example, doctors or clinically active scientists can be presented with the relevant information through the use of dashboards and queried repeatedly with little effort.