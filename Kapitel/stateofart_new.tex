Exploring and analyzing clinical data with visualization tools is already a known research topic~\cite{Caban.2015, Sittig.2015, UnberathPhilipp.2019, Festag.2019}. Festag et al. used KNIME analytics platform to implement a workflow for the MIMIC-III Data and the related MIMIC-III Waveform Database especially for time-series data~\cite{Festag.2019}. The user interface was implemented through so-called meta-nodes to set up the connection to both used databases in a first step~\cite{Festag.2019}. Next, a user was able to enter one or more subject IDs (representations for patients in the DB) and chooses between data sources~\cite{Festag.2019}. As the last step, the user decides which of the time series items or signals should be visualized with a line-plot ~\cite{Festag.2019}.
Another approach for MIMIC-II DB (previous version of MIMIC-III DB) was made by Lee et al. who established a data visualization tool based on web-based tools like HTML, CSS, and JavaScript libraries (jQuery and D3.js) and tools on the server side like PHP and PostgreSQL~\cite{Lee.2016}. The tool was designed to give the user a quick visualization of key aggregate statistics of the DB like sample size, distributions of clinical variables for a selected cohort. Therefore two main features were implemented, namely a explore- and a compare feature ~\cite{Lee.2016}. The Explore feature enables the user to select first a patient cohort based on variables like admission ICU service type, gender, age, and primary International Classification of Diseases 9 (ICD-9) code. In the next step further subselection was provided by the possibility to filter information like administrative information, demographic information, Interventions, lab test results, patient outcomes, vital signs and other miscellaneous variables provided by the MIMIC-II data~\cite{Lee.2016}. The Compare feature enables the user to visually compare two patient cohorts for
selected variables to be visualized~\cite{Lee.2016}. Visualization of time series data for one patient was not implemented by this project. Both attempts are promising regarding the visualization of clinical data, but setting up a hole web space with all the needed programming skills can be time-consuming and hard to maintain while handling visualizations with KNIME can be tricky and not the first choice when you think of interactive visualizations. Tableau is easy to use as a drag- and drop interface with a strong visualization capability that could meet all the requirements we identified for a tool that supports easy to use interdisciplinary data-driven research.