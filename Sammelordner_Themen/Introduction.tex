Comprehensive data collections are indispensable for answering complex scientific questions. In health and medical research, the collection of electronic patient records together with multi-omics data (genomics, epigenomics, transcriptomics, proteomics, metabolomics, interactomics, pharmacogenomics, diseasomics) results in heterogeneous databases that can offer enormous potential for clinicians, researcher and administrators due to the interconnection. 

%%%
Due to the widespread use of relational databases, the most common form of how data is represented and stored today is probably in the form of multiple tables linked via primary and foreign keys. In this way of data storage and presentation, the diverse relationships between data values and types, which are so important to recognize hidden unknown relationships, are only implicit apparent and must be illustrated with the help of additional tools. Visualization tools such as Tableau, Knime or Excel can be helpful in providing exploration and analysis of the data even for a group of users without explicit programming experience, but they also require a certain prior knowledge of the underlying data structure before the analysis can be started. 

Graph databases could allow in addition to the pure data storage additional visual access to the data. Due to the implementation, graphic databases are already designed for visual dynamic interaction and enable even inexperienced users to access the data and structures. easy, individual expandability of existing tools, even for typical users who are not experts in statistics or data mining. 

The Medical Information Mart for Intensive Care (MIMIC) is an example of such a large clinical database (DB) containing multiple tables with data of patients who were treated at intensive care units of the Beth Israel Deaconess Medical Center (Boston, Massachusetts, USA) between 2001 and 2012 ~\cite{Johnson.2016}. The third version MIMIC III collects clinical information, about more than 40 000 patients whit information about clinical measurements, administrative processes, laboratory test results, (pharmacological) therapy descriptions, and caregiver notes. ~\cite{Johnson.2016}. To protect the patient's personal information, an online course must be completed in advance and a data usage agreement must be accepted by the researchers who wish to access the data.



%%%%%%%%%%%%%%%%%%%%%%%%%%%%%%%%%%%%%%%%%%%%%%%%%%%%%%%%%%%%%%%%%%%%%%%%%%%%%%%%%%%%%%%%%%%%%%%%%%%%%%%%%%%%%%%%%%%%%%%%%%%%%%%%%%%%%%%%%%%%%%%%%%%%%%%%%%%%%%%%%%%%%%%%

In addition, this data may have to be linked together using multiple joins in advance of the actual analysis. 

The field of visul The amount of data in the healthcare system will continue to grow ~\cite{Murdoch.2013}. 

Representing data in a visually appealing way is not only nice to have, but the way data is represented can also change the perspective and understanding  lead to 

Visual analytics is an emerging discipline that has shown significant promise in addressing many of these information overload challenges.


One goal of visual analytics is to find hidden relations in the data and turn this findings into useful knowledge.